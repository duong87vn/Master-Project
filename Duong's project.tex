\documentclass[12pt]{article}
\linespread{1.3}
\usepackage{amsmath}
\usepackage[inner= 1cm,outer=1cm]{geometry}
\usepackage{graphicx}
\usepackage{dirtytalk}
\title{Master Project - Draft }
\author{Duong Than}
\date{\today}
\begin{document}
\maketitle	
	
		\section{Introduction :} 
		\subsection{Background} 

			Public health officials are interested in finding hotspots where disease risks is unusually high. They want to immediately identify any disease clusters and also potential disease hotspots in order to implement prevention programs before the outbreak turns into an epidemic. \\ 		
			
			Consequently, researchers have developed different methods to detect disease clusters, and there are many ideas that used developed point and regional count data to come up with an optimized solution. \\
			
			\textbf{What is a hotspot or a cluster?} A disease cluster defined as a collection of connected regions that has an unusually high proportion of observed cases to expected cases.Practically, there are more cases in those regions than we would expect if disease risk is constant over all regions.\\
			
			\textbf{An example of an appearance of disease clusters:} There is a state where certain counties  have an abnormally high number of observed cases. Public health officials worry about the uncontrolled spread of this disease, so they want to know exactly where the observed cases are significantly high and where there could be the hotspots. A comparison between the observed cases and the population of each region is made. So our ideal solution is to find some zones (could be more than one connected regions) where the observed cases are significantly large, and the rest of the zones are not significantly high. Ideally, we would consider all connected zones as a hotspot. However, if the number of regions is large, it is not feasible for us to be able to find all the connected regions within the study area. Instead we need to come up with clever ways to find some \say{reasonable} subsets of regions to consider as potential hotspots. Furthermore, with respect to statistics, there are some restrictions for our zones. Since we do not want to consider a zone with many regions and with a very large population,each zone has a limit for the number of regions and the total population of each zone has to be less than or equal a half of the entire study area's population. These restrictions help us to reduce the size of possible zones of the entire study area. However; datasets can be varying in many ways, we still need to come up with more clever methods of finding appropriate zones to detect for clusters. \\       
			
			\textbf{Objectives:} One of the main purposes of this paper is to compare and contrast three existing methods for detecting hotspots. Also we want to approach this problem from a graph theory point of view. In addition to this, we will propose some improvements for these methods. But first, we want to introduce the type of data we will consider. \\
				
		\subsection{Data Structure:} 
			In this section, we go into more detail about the data typically observed and standard assumptions about the dataset. 
			The basic form of the data involves a set of \textit{counts observed} (one count for each region ) and a matching set of \textit{counts expected} reporting the number of cases we expect in each region, under the null hypothesis that everyone in the entire study area has the same constant risk. \\
			
We often assume that the data represent a set of counts arising from a \textbf{Heterogeneous Poisson process}.
In particular, many methods model the regional counts as independent Poisson random variables based on one of the basic properties of a spatial Poisson process: event counts from non overlapping regions follow independent Poisson distributions based on an underlying intensity function defining the expected values (and variances). Also, the counts are non negative and discrete, which are the two main properties of Poisson distribution. \\
			 
			
			
			As an example of the structure of a spatial dataset of regional counts, we consider a Leukemia dataset for 218 regions in New York. This dataset contains 218 observations related to leukemia cases in 8 counties in the state of New York. The data were made available in Waller and Gotway (2005) and details are provided there. In this paper, we study geographical regions and their connected neighbors.		
The graphics below is illustrats a study area and the regions in it. For example, suppose we want to look at center region 83, and we want to study its five nearest neighbors. The algorithm to determine the nearest neighbors is to use the Euclidean distance to find the five centroids that have closest distances to centroid 83. Thus the five nearest regions of region 83 which are 89,90,84,88, and 75. We successively consider region 83 and its five nearest neighbors in Fifure: \\	
			
		\begin{tabular}{|c|c|c|}
			
		\hline
			\includegraphics[scale=0.2]{nyplot.png}
			  & \includegraphics[scale=0.2]{ny83.png}
					&\includegraphics[scale=0.2]{ny89.png} \\ \hline
						\includegraphics[scale=0.2]{ny90.png}
							& \includegraphics[scale=0.2]{ny88.png}
							   & \includegraphics[scale=0.2]{ny84.png} \\ \hline
									\includegraphics[scale=0.2]{ny75.png} && \\
		\hline	
			%\includegraphics[scale=0.2]{centroid.png}	\\
		\end{tabular}	\\
			The data $Y_1,Y_2,\dots,Y_N$ in regions $1,2,\dots,N$ are mutually independent Poisson random variables, where $Y_i$ are the observed cases in region $i$. \\ 
			The associated population of each region is $n_1,n_2,\dots,n_N$. In this problem, we want to use the counts observed in each region to determined where the risk is unusually high.\\
Table below is an example of a dataset, which include region 83 and its five nearest neighbors, was extracted from the Leukemia New York dataset: \\

		\begin{tabular}{|c|c|c|c|}
		\hline
		Region & Population $(n_i$)& Observed Cases $(Y_i)$ & \\ 
		83 & 5532& 3.33995 & \\
		89 & 2921& 8.1795 & \\
		90 & 3711& 5.22804 & \\
		84 & 2592& 1.15928 & \\
		88 & 3242& 2.19922 & \\
		75 & 4189& 0.50936 & \\
		\hline
		\end{tabular}\\
			
			\subsection{Method:} 

			For the most part, methods to assess clusters and clustering in count data assume some background information on the entire population rather than a set of controls. 
			We divide the entire study area into N regions $ i = \{1,2,\dots,N\}$. \\
			Since we consider the population sizes $n_1,n_2,\dots,n_N$ fixed ,the total population in the study area, \\
\[
n_+ = \sum_{i =1}^{N}n_i,
\]
is also fixed. \\
Let $Y_+ = \sum_{i=1}^{N} Y_i$. If we condition on $n_+$, the total number of observed cases, we get the probability $ r = \dfrac{Y_+}{n_+}$, which is a fixed constant risk for all the individuals under the constant risk hypothesis (CRH).\\
		
			\textbf{Test Hypothesis :} \\ 
			Let $Z$ be a collection of connected regions which we call a zone. \\
			We define $Y_+(Z) = \sum_{i \in Z} Y_i$ , $n_+(Z)=\sum_{i \in Z} n_i$ and $E(Z) = r*n_+(Z)$. \\
			
			\textit{Null hypothesis :} The number of observed cases for each region equals the expected number of cases. \\ 
			\begin{center}
			$H_0 : r_i = r$ for all $i$
			 \end{center}
			 where $E_i$ are the null expected number of cases in region i. Also, $E_i = r * n_i$.\\
			
			
			\textit{Alternative hypothesis :} There is at least one zone for which the underlying risk is higher when compared to the null hypothesis. \\
 			\begin{center}
			$H_1$ : $r \geq r$. \\
			\end{center}	
			With the assumption of Poission Process, below is the test statistic of likelihood ratio test for spatial clustering developed by Kulldorff : \\
				\[
					T = \max_{z\in Z} \bigg[\dfrac{Y_+(Z)}{E(Z)}\bigg]^{Y_+(Z)} \bigg[\dfrac{Y_+(Z^c)}{E(Z^c)}\bigg]^{Y_+(Z^c)} I\bigg[\dfrac{Y_+(Z)}{E(Z)} > \dfrac{Y_+(Z^c)}{E(Z^c)}\bigg]
				\]	  (1)
		
				where $ Z^c$ denotes all the regions outside zone Z, Y() denotes the observed cases within a zone, and $I()$ is the indicator function. \\ 
			
			 \textit{Commentary}: A zone is considered as cluster when its T-value is high enough so that it has to be statistically significant at the 0.05 level. Note that different techniques of picking potential clusters resulted in different T-values and T-value thresholds which determine whether clusters are statistically significant. \\
			 
			 \textbf{To determine significance:} \\
			 \begin{enumerate}
			 \item Calculate the test statistic of the observed data.	
			 \item Re-calculate it using a specified number (eg, 99, 499, etc.) of simulated data sets or permuation. This calculation is used to generate the expected distribution of the test statistic under the null hypothesis. 
			 \end{enumerate}
			 If A zone whose T-value falls into the $5\%$ of the extreme values of the expected distribution of the test statistic, it is statistically significant. \\
			
			\textbf{Restrictions for a target zone : } As we mentioned earlier, philosophically there are some restrictions for clusters. For each region's centroid , we determine $k$ (nearest) neighbors such that : \\
				\begin{enumerate}
					\item The farthest nearest neighbors from the centroid is less than or equal the max distance $d$ which depends on the dataset. \\ 
					\item The population of zone is less than or equal a half of the entire study area's population. ($n(Z) \leq 0.5(n+)$).\\
				\end{enumerate}
			
				\subsection{The Problem In a Graph Theory Framework} 
				
				Since we are working with spatial data, it is interesting to translate this problem into a graph theory problem from which we can use some graph theory techniques to find clusters. 
				%A centroid of each region is picked at the location where it has the biggest density of population of the region. 
				
				Firstly,we want to introduce the problem in graph theory's terminology. We let the entire study area be a graph $G$ where : \\
				\begin{enumerate}
					
				\item $V(G)$ is the set of all the centroids. Each vertex of $G$ is a centroid. \\
				\item $E(G)$ is the set of all the two centroids which represent for a pair of regions that share their boarders together. Each edge of $G$ represents for a pair of regions that share boarder together.\\
				\end{enumerate}
				
				Each vertex of $G$ contains information of each region that it represents. For instance  each vertex contains observed cases, expected number of cases, and population.
				
				A cluster is some induced connected subgraph $G[H]$ that would give us the maximum value of the test statistic test in(1). Regardless of which method is used, a cluster is an induced connected subgraph of $G$ and has at least following properties : \\
				\begin{enumerate}
					
					\item $|H| \leq k $ where $H$ is the set of vertices of this induced connected subgraph.
					%\item $ \forall v_i \in  V(G[H]), d(v,v_i) \leq d $
					\item $n(H) \leq 0.5(n_+)$, where $n(H)$ is the total population of vertices/regions $v_i \in H$. 
				\end{enumerate}
				Also, depending on different methods, there maybe additional constraints for induced connected subgraphs to be considered clusters.\\ 	
			
				Combinatorially, there are total $2^ {\binom{N}{2}}$ induced connected subgraphs in $G$. When $N$ is large, it becomes more difficult to search for clusters within this many number of possible zones. In theory, we are able to find all induced connected subgraphs of a given graph in order to find true clusters. However, when the graph is large enough, it is not feasible to do so. Thus, the question is: How do we find a clever way to search for the induced subgraph that maximizes the test statistic. This is the reason why researchers have proposed different methods for this question. The next section will introduce us to the three existing methods for finding clusters in a reasonable running time of searching.\\
			
			
\section{Existing Methods :} 
			
				We consider three different spatial scanning methods that have proposed : Circular Scanning Test, Upper Level Set, and Flexible Scanning Test. ALl of these methods use the likelihood ratio test to determine which regions could be potential disease clusters using a particular spatial dataset. However, the approaches use different ways of forming sets of zones to be tested. We describe each method in more detail.\\			
				\begin{enumerate}
			\item \textbf{Circular Scanning Test} : Kulldorff detected potential clusters (hotspots) for the study area by making circles whose center is the centroid of each region $i$.Iteratively increassing the circle radius to include new regions until necessary constraints are no longer satisfied. 
	
			\begin{enumerate}
				\item Let region $i$ be a zone and calculate the test statistic. 
				\item Extend the zone to include the nearest neighbors of regions $i$ not included in the zone and compute the test statistic for the new zone. 
				\item Repeat b until the restrictions mentioned in section 1.3 are no longer satisfied.  
				\item Applying those steps to every single region of the entire study area to find all the possible candidate zones. 
			\end{enumerate}  
		In this method, at each region, we conduct a search around it, and the scanning window can contain a maximum of $k$ regions in it. Thus the total number of possible candidate zones is $N(k)$ which is feasible to conduct a search.
		As a result, the zones with highest T-value, and that are statistically significant within each scanning window are considered clusters. \\
			
			There are some advantages to this method. We search for potential clusters using every single region. We can find all the potential zones in a linear time. For each region we start the search, there are maximum $k$ zones as a result. Thus, there are maximum $N(k)$ running times. However, there are some disadvantages we encounter. We define a hole is a region that has no observed cases. Firstly, there would be holes in zones. Since we determine our potential zones using geographical distance, some holes might be included in some zones, which will increase the t-value of those zones, but are not hotspots. Secondly, the shapes of the zones are most likely to be circles, whereas geographical zones can be in more complicated shapes. Thirdly, since we keep increasing the radius of the zone until it meets one of the restrictions, we might neglect a single region as a potential zone itself. \\  
		From a graph theory point of view, to find the potential zones, we start from a vertex. From a vertex $v$, we find the sequence induced connected subgraph $G[H]$ with the following properties : \\
				\begin{enumerate}
					
					\item $|H| \leq k $ 
					\item $ \forall v_i \in  H, d(v,v_i) \leq d $
					\item $n(H) \leq 0.5(n_+)$ where $n(H)$ is the total population in centroids $v_i$. 
				\end{enumerate}   
				 For each time we increase the distance from $v$, the test statistic is computed for each induced subgraph. This is repeated for each vertex. \\ 
				
	
	The figure below is an demonstration of the scanning process of the Circular Method where region 11 is the original region of the search:\\
	\includegraphics[scale=0.45]{demo_1} \\
	Start with the centroid at region 11., we increase the radius up to 1.5 unit. For the radius of 1, regions 10,12,5,17 are the nearest neighbors of region 11.\\
	Then we increased the radius up to 1.5, regions 4,6,16,18 are the nearest neighbors of region 1.\\
	In this example,three candidate zones that are formed: \\
	\begin{enumerate}

	
	\item Candidate zone 1: region 11\\
	\item Candidate zone 2: Regions 11,5,10,12,17\\
	\item Candidate zone 3: Regions 11,5,10,12,17,4,6,16,18\\
	
	\end{enumerate}
	 
	 \item  \textbf {Upper Level Set (ULS) Scan Statistic :} \\
			The ULS scan statistic is an adaptive approach in which the parameter space is reduced by using the empirical cell rates,
\[
			 \hat{r_i} = \dfrac{Y_i}{n_i}
\]
			They define the Upper Level Set is the set that comprises all the regions with each level $g$.\\
			\[
U_g = \{ i : \hat{r_i} > g\}
			\]
 Within each Upper Level Set, they form their target zones as the following: \\ 
				
				Let $ULS= \{1,2,\dots,t\}$ be the Upper Level Set of regions $i$ such that $\hat{r_i} \leq \hat{r_j}$ if and only if $i < j$.  
				\begin{enumerate}
					\item Region $1$ is the first zone $Z_1$. 
					\item If region $2$ connects to region $1$, then $2 \in Z_1$. If region $2$ does not connect with region $1$, then $2$ is $Z_2$. 
					\item Keep repeating this process until we run out of regions in the ULS. \\   
				\end{enumerate}
			 In this method, the searching is reduced down to only $t$ numbers of regions instead of $N$.The total number of possible zones is less than or equal $t$ which is feasible to conduct a search. And a cluster is a zone within $U_g$ that has the highest T-value and is statistically significant compare to other zones within set $U_g$.\\ 
		An advantage of this method is the running times while searching is much smaller. There are maximum $t(k)$ times and $t \leq N$. \\	
		However, there is a disadvantage that we do not take consideration of all regions. Some regions with small $G_i$ values that give large $T-values$. \\
		
		If we look at this method under graph theory's point of view , a ULS set is a simple planar graph $H$. And our potential zones are the components $\{C_i\}_{i=1}^{t}$ of $H$. Also, each component $C_i$ has the same properties as each induced connected subgraph $G[H]$ in the circular scanning test. \\  
				
	
	The figure below is a demonstration of the scanning process of the ULS method. We set $g$ equals the mean of the observed cases of the entire study area.\\
	\includegraphics[scale=0.5]{demo_2} \\
	Red regions are regions whose $G_i$ is greater than the mean of the observed cases. As we can see that there are two different candidate zones that we need to look at. Now the total number of regions that we want to search for clusters are only 9 regions instead of 36 regions. This step reduces a significant times of searching for clusters. 
		\item \textbf{ Flexible Scanning Test:}	\\
			
			The flexible scanning test is a different method to pick up potential zones for clusters. In this method,the scanning window of each region i comprises a maximum k regions which includes region i and k-1 nearest neighbors of region i. Within these k regions, a cluster is the zone whose included regions are a subset of the k regions that is connected and has the highest t-value compare to other zones of these k regions. Thus, the potential zones can have a different shape rather than a circle which the author of this method claimed that this method is better than the circular method. \\
			
				Let $Z_{ik}$ denotes the window composed by the nearest ${k-1}$ neighbors to the region i.Thus all the windows are scanned by the spatial scan statistic are included in the set : \\
		\[
			Z = \{Z_{ik} | 1 \leq i \leq N, 1 \leq k \leq K \}
		\] 
				In this method, there are maximum $ N\sum_{i=1}^{k} i $ number of regions that are searched, which makes the clustering searching more feasible. However, $ N\sum_{i=1}^{k} i $ is quite a small number of zones compare to possible $2^{\binom{n}{2}}$ zones we could have in this study area. This raises a concern whether this method is sufficient in searching for clusters or not. \\
			
			In graph theory's language, clusters in this method are induced connected subgraphs of $G$ which have less than or equal k number of vertices. \\
			 	
		% Under the scanning test, our null and alternative hypothesis are : \\
			
		%	$H_0 : E(N(Z)) = a(Z)$ for all Z .\\
			
		%	$H_1 : E(N(Z)) > a(Z)$, for some Z. (There is at least one window Z for which the underlying risk is higher inside the window when compared outside.)\\
		%		where $N()$ is the random number of cases, and $ a()$ is the null expected number of cases. \\
	
		%	Under the Poission assumption, the test statistic, which was constructed with the likelihood ratio test, is given by : \\
				
		%		\[
		%			sup_Z(\dfrac{n(Z)}{a(Z)})^{n(Z)} (\dfrac{n(Z^c)}{a(Z^c)})^{n(Z^c)} I(\dfrac{n(Z)}{a(Z)} > \dfrac{n(Z^c)}{a(Z^c)})
		%		\]	 
	
		%		where $ Z^c$ denotes all the regions outside window Z, and n() denotes the observed cases within the specified window, and I() is the indicator function. \\ 
	The figure below is a demonstration of the scanning process of the Flexible method. Again, the original region is 11, and we set the maximum number of regions for each zone is k=7.\\
	\includegraphics[scale=0.5]{demo_3} \\
	In this demonstration, we choose region 11 as the center region, then pick the first nearest region to 11 is 10,12,5,1. After all then first nearest regions are picked, the second nearest regions are picked and so on until we reach the maximum k. Thus,16 and 18,the second nearest regions to 11 are picked. Note that in this example, we can also pick 4 and 6 instead of 16 and 18. And this could increase the number of searching times. However, in real life, it is almost impossible for two regions to have the same distance to a third one. \\  
	
	In this example, we generated 14 candidate zones which are: \\
	
	
		
	
	\begin{tabular}{|c|c|}
		\hline
		Candidate zone & Regions  \\
		\hline
		1 & 11  \\
		2 & 11,5 \\
		3 & 11,10 \\
		4 & 11,12 \\
		5 & 11, 17 \\
		6 &  11,5,10 \\
		7 & 11,5,12 \\
		8 & 11,10,17 \\
		9 & 11,12,17 \\
		10 & 11,12,17,18 \\
		11 & 11,10,16,17 \\
		12 & 5,11,12,17,18 \\
		13 &  5,11,12,16,17 \\ 
		14 &   5,10,11,12,16,17,18 \\ 
	\hline
	\end{tabular}
	
	\section{A Contrast of the Three Methods Applied on the Leukemia New York Data}
	In this section, using the Leukemia New York data, we take a closer look at the differences among these three methods in determining their own clusters. \\
	The table below shows clusters of each method:\\
	
	\begin{tabular}{|c|c|c|c|}
	\hline
	& Circular & ULS & Flexible \\
	
	 & \includegraphics[scale=0.3]{cluster_circular} & \includegraphics[scale=0.2]{cluster_uls} & \includegraphics[scale=0.2]{cluster_flexible}\\
	\hline
	\end{tabular}
	
	 \subsection {Circular Method} 
	
	 \begin{tabular}{|c|c|c|c|c|}
	 \hline
	 & Observed Cases & Population & Log of T-value& P-value  \\
	Cluster 1: & 117 & 135295 & 15.00556 & 0.01\\ 
	Cluster 2:& 47 & 48501 & 7.851015 & 0.09\\
	Cluster 3:& 44 & 45667 & 7.199672 & 0.1\\
	\hline
	\end{tabular} \\
	
	\subsection{Upper Level Set Method} 
	
	 \begin{tabular}{|c|c|c|c|c|}
	 \hline
	 & Observed Cases & Population & T-value &P-value  \\
	Cluster 1: & 78 & 66679 & 21.61002 & 0.01\\ 
	Cluster 2:& 73 & 62749 & 19.86723 & 0.03\\
	% Cluster 3:& &  &  &\\
	\hline
	\end{tabular} \\
	
	\subsection{Flexible Method} 
	
	 \begin{tabular}{|c|c|c|c|c|}
	 \hline
	 & Observed Cases & Population & T-value & P-value  \\
	Cluster 1: & 39 & 31420 & 11.67128& 0.01\\ 
	Cluster 2:& 43 & 36446 & 11.64168 &\\
	 % Cluster 3:& &  &  &\\
	\hline
	\end{tabular} \\
	
	
	\textbf{Circular Scanning Method:} Three clusters have circle-like shapes, and each cluster includes more regions than the other two methods.\\
	\textbf{ULS Scanning Method:} The shape of the two clusters are different than circles. \\
	\textbf{Flexible Scanning Method:} The maximum number of regions for each cluster is restricted at 15.	
	The shapes of clusters in this method are more flexible than the circle shape in Circular Scan Method. These clusters are overlap with clusters in Circular method except it has only two instead of three clusters. \\
	
	\section{A Small Example}
	Although we have an access to the Leukemia dataset, it is easier to contrast the three methods using a smaller scale of dataset and the true clusters are known. This section will help us to see in a smaller example how each method pick up its clusters. So we created a dataset of a study area which had 36 regions. The spatial setup of this dataset is a $6\times6$ grid. The population for each region was randomly generated using Poisson distribution with the mean of 1 million. We picked in advance 2 true clusters. One cluster had 4 connected regions and the other one had 3 connected regions. The observed cases for each region was generated as the following: \\
		
	\begin{tabular}{|c|c|}
	\hline
	Regions & Observed Cases \\
	\hline
	7,11,12,17(Cluster 1) and 20,26,32(Cluster 2) & Observed cases = 0.005 * population \\ 
	4,5,10,14,15,18,24,23 & Observed cases = 0.002 * population \\
	1,2,3,6,8,9,13,19,21,22,25,26,28,29,30,31,33,34,35,36 & Observed cases = 0.001 * population \\
	\hline
	\end{tabular}	
	
	\end{enumerate} 
	
	This is the layout of our study area for our small examples:\\
		\includegraphics[scale=0.3]{Area_layout} \\
	
	And the figure below shows the density of the observed cases of each region in the entire study area.\\
	\includegraphics[scale=0.2]{density_cases}\\
	
	As we deliberately determined the two true clusters, the closest picture of our study area should have looked like the below graph: \\
	
	\includegraphics[scale=0.25]{true_clusters}\\ where the connected green nodes were divided into the two separating clusters. \\
	
	Then we tested the three methods using this dataset. \\
	
	\subsection{Clusters of Circular Test } 
	 \includegraphics[scale=0.3]{test_1} \\ The circular method produced two clusters. One had 6 connected regions, and the other one had 3 connected regions. This method detected one true cluster. and it also detected the second true cluster but added two extra region in it. \\
	\subsection{Clusters of ULS Test}
	 \includegraphics[scale=0.35]{test_2_01} \\ We found these two cluster using ubpop = 0.1. There were two different clusters as the result. Both clusters comprised three connected regions. But when using ubpop* = 0.2 the two clusters were slightly different which we can see in the below figure:\\
	 
	 \includegraphics[scale=0.3]{test_2}\\ This result gave us exactly two clusters that we deliberately picked. Thus this method detected the right hot spots when we used an appropriate ubpop.  \\ 
	\textit{ubpop:} a proportion between 0 and 1 containing the upper bound for the proportion of total population contained collectively among a set of nearest neighbors.\\
	\subsection{Clusters of Flexible test} 
	 \includegraphics[scale=0.4]{test_3} \\ We found these two clusters when we ran this test with k=4 which were the true clusters.\\ 


\subsection{Conclusion}

In this small example,ULS and Flexible methods detected the true clusters with an appropriate setup of some parameters while the Circular method detected two extra regions to be considered as high risk spots. These results are compatible with the New York Leukemia data with respect to the size of clusters. In section 4, we found that the Circular method detected bigger size of clusters comparing to clusters of the other two methods. Thus we might suspect that the Circular method tends to detect more false hot spots than the ULS and Flexible methods. A possible explanation for this observation is in Circular method, candidate zones are in circle shape. Thus, a detected hotspot could include a few false regions if those are in the same zone with other true regions.  

To strengthen our suspicion, we tested another example with true clusters that had different shapes than our first example. We wanted to see if the results are consistent with our assumption about Circular method.\\ 
The following figure are shown the two true clusters.\\
\includegraphics[scale=0.2]{ex2_true}\\

\hspace{4cm}\begin{tabular}{|c|c|}
	\hline
	Cluster & Regions \\
	\hline
	1 & 19,21,26,31,33 \\
	2 & 5,10,11,17 \\ \hline
\end{tabular} \\

First, we found 3 clusters using Circular method:\\
\includegraphics[scale=0.2]{Ex2:Circular} \\

\hspace{4cm}\begin{tabular}{|c|c|}
	\hline
	Cluster & Regions \\
	\hline
	1 & 19,25,26,31,32,33 \\
	2 & 5,10,11,12,17 \\ 
	3 & 21 \\ \hline
\end{tabular} \\

This results had not surprised us as clusters included three extra regions compare to our true clusters.\\

The next figure illustrates the clusters of ULS method: \\ 
\includegraphics[scale=0.2]{Ex2:ULS}\\
With ubop = 0.3, the ULS method picked up the right true clusters.\\

And this is the results using Flexible method:\\
\includegraphics[scale=0.2]{ex2:Flexible}\\
In this example, in order for Flexible method to pick up the right true clusters, we need to increase $k$ value up to 9 which unexpected since the largest size of our true clusters was five. \\

So there are trade offs between using Circular and Flexible methods. The technique of selecting candidate zones of Circular method and Flexible method are similar since they tend to pick the nearest neighbors of a given center zone. However, using Flexible method, we have more candidates zones, and we can find the true clusters with less risk of including false regions. One withdraw of Flexible method is that it can be difficult to computationally run Flexible method when the study area is big enough. On the other hand, Circular method can help us to search for clusters more quickly, even though we might include some false regions in the results. \\
   
The ULS method has the least number of candidate zones. Thus it is not quite comparable to compare and contrast to other two methods. However in our small example, this is the most effective scanning method in searching for the true clusters. \\



\section{Suggestions for Improvements}
\section{Glossary}
\subsection{Technical Terms}
\begin{tabular}{|c|c|}
\hline
\textbf{Statistical Terminology }& \textbf{Graph Theory Terminology} \\
\hline
A Study Area & A Graph $G$ \\
A Region & A Vertex \\
A boarder & An edge \\
A Zone & An Induced Connected Sub-graph\\

\hline

\end{tabular} \\

 	
\subsection{Definitions:} 
\begin{itemize}
\item True region:  region that belongs to the true clusters. \\ 
\item False region:  region that does not belong to the true clusters. \\ 
\item Candidate zone:  one or more connected regions that is formed while a method is searching for clusters.\\ 
\item Candidate pool:  a pool that contains all the candidate zone of a method when applied to a specific study area. \\
\end{itemize}
\end{document}
